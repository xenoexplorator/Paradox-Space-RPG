\chapter{Introduction}

\section{Playing the game}

Unlike many other role-playing games, \ParadoxSpaceRPG is not played with dice.
You will instead need a standard deck of playing card --- 4 suits of 13 cards each, for a total of 52.

Whenever the character you are playing attempts an action whose result is uncertain, you draw a card to
determine the outcome of your endeavors. Meanwhile, the \GM sets an appropriate \TN.
The value of the card drawn and your character's abilities both contribute to the total value of your
draw, which must equal or exceed the \TN in order for your action to succeed.

When you draw a card to resolve an action, your character may be able to apply one of his or her Aptitudes
(as noted on the relevant character sheet). In those circumstances, the rank of your aptitude is added
to the value of your card before checking against the \TN. If your card's suit matches that of your Aptitude,
you may also opt to draw a new card to use instead of your first one.

If none of your character's Aptitudes is applicable to the action you are attempting, your character
is Winging It. Winging It adds $2$ to the value of your card, but without providing opportunities for a redraw.

The value of a card is the number appearing on its face (an Ace has a value of $1$). Figures (the Jack, Queen
and King) have a value of $11$, $12$ and $13$, resepectively.

Once an action has been resolved, used cards are placed face-down in a discard pile next to the drawing deck.
Participants (including the \GM) are not allowed to look at the cards in either pile.
Should the draw deck be emptied during a gaming session, simply refill it by shuffling the discard pile.
The first ten cards of the deck should then be moved into the discard pile.

\section{Material}

In addition to this manual (in either print or digital format), you will need the following material to play
\ParadoxSpaceRPG.

\begin{itemize}
\item Paper
\item Pens, pencils, or other writing utensils
\item A deck of playing cards
\end{itemize}

When setting up for a game session, shuffle the deck of cards before placing it in the center of the gaming area
(where it can easily be accessed by every participant), forming the drawing deck. THe discard pile is then formed
by taking, face down, the first ten cards of the drawing deck. This adds uncertainty when drawing cards, as it
makes it impossible to know which cards have already been taken out.
