\chapter{Characters}

Player characters creation rules, maybe some basics on how to play the game

\section{Who are you?}

Start by conceptualizing your character.
Name, online nickname, age, relations to other players

\section{What can you do?}

Your character's ability to successfully complete tasks is based on his or her \gameterm{Aptitudes}.
Your character's Aptitudes describe activities that are of some import to him or her, whatever the
reason may be. As such, Aptitudes are split into four categories:
\begin{itemize}
\item \gameterm{Passions} describe activities your character is engaged in and performs for their own sake.
When drawing for an action related to a Passion, you can opt to redraw (using the result of the new card)
as long as you keep drawing hearts (\heart).

\item \gameterm{Talents} describe activities your character is naturally talented in and performs instinctively.
When drawing for an action related to a Talent, you can opt to redraw (using the result of the new card)
as long as you keep drawing diamonds (\diamond).

\item \gameterm{Skills} describe activities your character has learned to perform, developing their abilities
through practice. When drawing for an action related to a Skill, you can opt to redraw (using the result of
the new card) as long as you keep drawing clubs (\club).

\item \gameterm{Ambitions} descripe activities your character seeks to excel in and performs in hope of improving
themselves. When drawing for an action related to an Ambition, you can opt to redraw (using the result of the new
card) as long as you keep drawing spades (\spade).
\end{itemize}

Aptitudes of all categories are given a rating expressed as a bonus ranging from $+1$ (you are not very good
at that Aptitude) to $+8$ (you are simply the best there is at that Aptitude). When creating your character, you
are given $10$ points to distribute freely among his or her Aptitudes, subject to the following restrictions:
\begin{enumerate}
\item the highest rating one of your Aptitudes can have at the beginning of the game is $+4$; and
\item you cannot start with two Aptitudes rated $+4$.
\end{enumerate}

Aptitudes are not derived from a master list used by every character. When creating yours, you can simply
choose any action or activity as one of his or her Aptitudes, as long as you follow any guidelines set by
your Narrator. Aptitudes that inform your character's personality and interests are preferred to those chosen
for their utilitarian value. For instance, "Karate" would be a better Aptitude than "Hand-to-Hand Combat".

\section{What must you learn?}

The central story of \ParadoxSpaceRPG is that of coming of age into adulthood.

Think of what your character has to learn in order to progress on that journey.
Note three things of varying importance (minor, major, life-changing)
These are broad experiences that you think should come up during play and will guide your character's growth.
Broadly speaking, these should suggest events that your character will learn from. Going through them might
make him or her more mature, or let him or her learn more about the person he or she is -- perhaps even be
a catalyst for a dramatic personality change. The goal of this exercise is to develop an idea of the direction
in which in-game character development could take you during play.

Communication with your \GM is an important part of this character creation step. His or her input can enrich
your reflection, while your ideas can help him or her plan for your game's future developments.
