\chapter{Characters}

Player characters creation rules, maybe some basics on how to play the game

\section{Who are you?}

Start by conceptualizing your character.
Name, online handle, age, relations to other players

\section{What can you do?}

Your character's ability to successfully complete tasks is based on his or her \gameterm{Aptitudes}.

Aptitudes synthesize your character's core capabilities in solving problems, expressed as a bonus ranging from
$+0$ (you are not very good at that Aptitude) to $+8$ (you are simply the best there is at that Aptitude).
\ParadoxSpaceRPG{} makes use of three core Aptitudes:
\begin{itemize}
\item \gameterm{Brains} describes your character's ability to use logic and reason to deduce information and
	understant the world.
\item \gameterm{Brawn} describes your character's strength, endurance, overall fitness and ability to
	accomplish physical tasks.
\item \gameterm{Beauty} describes your character's proficiency in usind their natural charm and social skills
	to convince others to help them.
\end{itemize}
When creating your character, you are given $10$ points to distribute freely across all three Aptitude. Your
character's starting Aptitude can only have a rating of $+4$ or less, however, as your character has yet to
fully realise hir or her potential (no matter what he or she may believe).

In addition to the core Aptitudes, your character is also defined by four \gameterm{Interests} that are of
great import to him or her. When your character performs an action related to one of his or her Interests and
you draw a card of the appropriate suit, you can opt to draw a new card and use that card's value instead.
\begin{itemize}
\item A \gameterm{Passion} is an Interest your character is engaged in and performs for their own sake.
	The suit of Passion is hearts (\heart).

\item A \gameterm{Talent} is an Interest your character is naturally talented in and performs instinctively.
	The suit of Talent is diamonds (\diamond).
When drawing for an action related to a Talent, you can opt to redraw (using the result of the new card)
as long as you keep drawing diamonds (\diamond).

\item A \gameterm{Skill} is an Interest your character has learned to perform, developing their abilities
through practice.
	The suit of Skill is clubs (\club).
the new card) as long as you keep drawing clubs (\club).

\item A \gameterm{Ambition} is an Interest your character seeks to excel in and performs in hope of improving
themselves.
	The suit of Ambition is spades (\spade).
\end{itemize}

\section{What must you learn?}

The central story of \ParadoxSpaceRPG{} is that of coming of age into adulthood. As your role is to interpret
your character through that story, it is important that you have some idea of what this transition will be like
for him or her.  Think of what your character has to learn in order to progress on that journey. Which
experiences does he or she already understand, and which lessons does he or she still have to learn?

Record three events of varying importance (one minor, one major and one life-changing) on your character sheet.
These are your character's personal \gameterm{Milestones}.

These are broad experiences that you think should come up during play and will guide your character's growth.
Broadly speaking, these should suggest events that your character will learn from. Going through them might
make him or her more mature, or let him or her learn more about the person he or she is -- perhaps even be
a catalyst for a dramatic personality change. The goal of this exercise is to develop an idea of the direction
in which in-game character development could take you during play.

Communication with your \GM{} is an important part of this character creation step. His or her input can enrich
your reflection, while your ideas can help him or her plan for your game's future developments.

\section{Going beyond}

As your character progresses through his or her adventures in Paradox Space, his or her growth will be shaped
through his or her experiences. This development is represented by players earning \gameterm{Advancement Points}
(\gameterm{APs}) at the end of each game session. The exact number of APs given to each player is determined by
the \GM. The following guidelines describe the pace of progression assumed by these rules.
\begin{itemize}
	\item $1$ AP for player attendance;
	\item $1$ AP for players who used their Interests in surprising ways;
	\item $3$ APs for players whose characters accomplish a minor Milestone;
	\item $7$ APs for players whose characters accomplish a major Milestone;
	\item $11$ APs for players whose characters accomplish a life-changing Milestone;
	\item Triple APs earned for accomplishing a Milestone defined during character creation.
\end{itemize}

APs are used to improve your character's Aptitudes in one of two ways:
\begin{itemize}
	\item Increase an Aptitude's rating by $1$ at a cost equal to the new rating.
	\item Change one of your character's Interest at a cost of $1$ AP.
\end{itemize}
